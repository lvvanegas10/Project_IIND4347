\documentclass{article}

\usepackage[utf8]{inputenc}
\usepackage{longtable}
\usepackage{authblk}
\usepackage{adjustbox}
\usepackage{natbib}

\title{LOS INDICES DE COLOMBIA}
% autores
\renewcommand\Authand{, y }
\author[1]{\normalsize Laura Valeria Vanegas García}

\affil[1]{\small  Facultad de Ingeniería,Universidad de los Andes\\
\texttt{{lv.vanegas10}@uniandes.edu.co}}

\date{29 de Junio de 2018}

\usepackage{Sweave}
\begin{document}
\Sconcordance{concordance:paper_version1.tex:paper_version1.Rnw:%
1 18 1 1 0 18 1 1 8 7 1 1 5 14 0 1 2 6 1 1 15 1 2 9 1 1 14 1 2 8 1 1 5 %
12 0 1 2 4 1 1 8 13 0 1 2 5 1 2 2 12 1 1 5 1 1 1 4 31 0 1 2 10 1 1 21 1 %
1 1 18 5 1 1 14 1 2 9 1}


\maketitle

\begin{abstract}
Este es mi proyecto final para el curso de verano IIN4347 de la Universidad ded los Andes para 2018-19. Este es mi primer trabajo en exploracion y modelamiento de indices usando LATEX. Este trabajo lo he hecho bajo la filosofía de trabajo replicable. Este es mi proyecto final para el curso de verano IIN4347 de la Universidad ded los Andes para 2018-19. Este es mi primer trabajo en exploracion y modelamiento de indices usando LATEX. Este trabajo lo he hecho bajo la filosofía de trabajo replicable.
\end{abstract}

\section*{Introducción}

En este documento presento mi investigacion sobre diversos indices sociales en Colombia. Los indices fueron extraidos de wikipedia y se realizó preprocesamiento en Python con ayuda de la librería Pandas. En este documento presento mi investigacion sobre diversos indices sociales en Colombia. Los indices fueron extraidos de wikipedia y se realizó preprocesamiento en Python con ayuda de la librería Pandas. 

En este documento presento mi investigacion sobre diversos indices sociales en Colombia. Los indices fueron extraidos de wikipedia y se realizó preprocesamiento en Python con ayuda de la librería Pandas.En este documento presento mi investigacion sobre diversos indices sociales en Colombia. Los indices fueron extraidos de wikipedia y se realizó preprocesamiento en Python con ayuda de la librería Pandas. En este documento presento mi investigacion sobre diversos indices sociales en Colombia. Los indices fueron extraidos de wikipedia y se realizó preprocesamiento en Python con ayuda de la librería Pandas.

Comencemos viendo que hay en la sección \ref{univariada} en la página \pageref{univariada}.

\clearpage

\begin{Schunk}
\begin{Soutput}
'data.frame':	32 obs. of  6 variables:
 $ X.U.FEFF.IDH      : num  0.879 0.867 0.865 0.849 0.842 0.839 0.837 0.835 0.834 0.832 ...
 $ Departamento      : chr  "Santander" "Casanare" "Valle del Cauca" "Antioquia" ...
 $ Población.Cabecera: int  1587972 281548 4169553 5262172 742812 761658 10070801 2438533 56487 506254 ...
 $ Población.Resto   : int  502867 93701 586560 1428858 539251 206109 914484 107391 21926 68756 ...
 $ Población.Total   : int  2090839 375249 4756113 6691030 1282063 967767 10985285 2545924 78413 575010 ...
 $ DepartamentoNorm  : chr  "Santander" "Casanare" "Valle del Cauca" "Antioquia" ...
\end{Soutput}
\end{Schunk}

\section{Exploración Univariada}\label{univariada}

Esta es la sección de exploración univariada. En esta sección exploro cada índice.Esta es la sección de exploración univariada. En esta sección exploro cada índice. Esta es la sección de exploración univariada. En esta sección exploro cada índice. 

Para conocer el comportamiento de las variables se ha preparado la Tabla \ref{Tfrecuencias}, donde se describe la distribución de las modalidades de cada variable. Los números representan la situación de algun país en ese indicador, donde el mayor valor numérico es la mejor situación.

% latex table generated in R 3.4.1 by xtable 1.8-2 package
% Fri Jun 29 15:09:13 2018
\begingroup\normalsize
\begin{longtable}{llrrr}
\caption{Tablas de Frecuencia de la variables en estudio} \\ 
 \textbf{Variable} & \textbf{Levels} & $\mathbf{n}$ & $\mathbf{\%}$ & $\mathbf{\sum \%}$ \\ 
  \hline \hline
Departamento & Amazonas & 1 & 3.1 & 3.1 \\ 
   & Antioquia & 1 & 3.1 & 6.2 \\ 
   & Arauca & 1 & 3.1 & 9.4 \\ 
   & Atlántico & 1 & 3.1 & 12.5 \\ 
   & Bolívar & 1 & 3.1 & 15.6 \\ 
   & Boyacá & 1 & 3.1 & 18.7 \\ 
   & Caldas & 1 & 3.1 & 21.8 \\ 
   & Caquetá & 1 & 3.1 & 25.0 \\ 
   & Casanare & 1 & 3.1 & 28.1 \\ 
   & Cauca & 1 & 3.1 & 31.2 \\ 
   & Cesar & 1 & 3.1 & 34.3 \\ 
   & Chocó & 1 & 3.1 & 37.4 \\ 
   & Córdoba & 1 & 3.1 & 40.6 \\ 
   & Cundinamarca & 1 & 3.1 & 43.7 \\ 
   & Guainía & 1 & 3.1 & 46.8 \\ 
   & Guaviare & 1 & 3.1 & 49.9 \\ 
   & Huila & 1 & 3.1 & 53.0 \\ 
   & La Guajira & 1 & 3.1 & 56.2 \\ 
   & Magdalena & 1 & 3.1 & 59.3 \\ 
   & Meta & 1 & 3.1 & 62.4 \\ 
   & Nariño & 1 & 3.1 & 65.5 \\ 
   & Norte de Santander & 1 & 3.1 & 68.6 \\ 
   & Putumayo & 1 & 3.1 & 71.8 \\ 
   & Quindío & 1 & 3.1 & 74.9 \\ 
   & Risaralda & 1 & 3.1 & 78.0 \\ 
   & San Andrés y Providencia & 1 & 3.1 & 81.1 \\ 
   & Santander & 1 & 3.1 & 84.2 \\ 
   & Sucre & 1 & 3.1 & 87.4 \\ 
   & Tolima & 1 & 3.1 & 90.5 \\ 
   & Valle del Cauca & 1 & 3.1 & 93.6 \\ 
   & Vaupés & 1 & 3.1 & 96.7 \\ 
   & Vichada & 1 & 3.1 & 99.8 \\ 
   \hline
 & all & 32 & 99.8 &  \\ 
   \hline
\hline
Población.Cabecera & 13090 & 1 & 3.1 & 3.1 \\ 
   & 17679 & 1 & 3.1 & 6.2 \\ 
   & 29009 & 1 & 3.1 & 9.4 \\ 
   & 34200 & 1 & 3.1 & 12.5 \\ 
   & 56487 & 1 & 3.1 & 15.6 \\ 
   & 69723 & 1 & 3.1 & 18.7 \\ 
   & 172650 & 1 & 3.1 & 21.8 \\ 
   & 179323 & 1 & 3.1 & 25.0 \\ 
   & 253058 & 1 & 3.1 & 28.1 \\ 
   & 281548 & 1 & 3.1 & 31.2 \\ 
   & 300012 & 1 & 3.1 & 34.3 \\ 
   & 506254 & 1 & 3.1 & 37.4 \\ 
   & 567393 & 1 & 3.1 & 40.6 \\ 
   & 570399 & 1 & 3.1 & 43.7 \\ 
   & 596651 & 1 & 3.1 & 46.8 \\ 
   & 714664 & 1 & 3.1 & 49.9 \\ 
   & 719730 & 1 & 3.1 & 53.0 \\ 
   & 742812 & 1 & 3.1 & 56.2 \\ 
   & 761658 & 1 & 3.1 & 59.3 \\ 
   & 775636 & 1 & 3.1 & 62.4 \\ 
   & 805262 & 1 & 3.1 & 65.5 \\ 
   & 907590 & 1 & 3.1 & 68.6 \\ 
   & 950107 & 1 & 3.1 & 71.8 \\ 
   & 967669 & 1 & 3.1 & 74.9 \\ 
   & 980694 & 1 & 3.1 & 78.0 \\ 
   & 1099363 & 1 & 3.1 & 81.1 \\ 
   & 1587972 & 1 & 3.1 & 84.2 \\ 
   & 1693659 & 1 & 3.1 & 87.4 \\ 
   & 2438533 & 1 & 3.1 & 90.5 \\ 
   & 4169553 & 1 & 3.1 & 93.6 \\ 
   & 5262172 & 1 & 3.1 & 96.7 \\ 
   & 10070801 & 1 & 3.1 & 99.8 \\ 
   \hline
 & all & 32 & 99.8 &  \\ 
   \hline
\hline
Población.Resto & 21926 & 1 & 3.1 & 3.1 \\ 
   & 27249 & 1 & 3.1 & 6.2 \\ 
   & 30356 & 1 & 3.1 & 9.4 \\ 
   & 43076 & 1 & 3.1 & 12.5 \\ 
   & 46106 & 1 & 3.1 & 15.6 \\ 
   & 49821 & 1 & 3.1 & 18.7 \\ 
   & 68756 & 1 & 3.1 & 21.8 \\ 
   & 93701 & 1 & 3.1 & 25.0 \\ 
   & 98058 & 1 & 3.1 & 28.1 \\ 
   & 107391 & 1 & 3.1 & 31.2 \\ 
   & 179573 & 1 & 3.1 & 34.3 \\ 
   & 196229 & 1 & 3.1 & 37.4 \\ 
   & 206109 & 1 & 3.1 & 40.6 \\ 
   & 241065 & 1 & 3.1 & 43.7 \\ 
   & 260411 & 1 & 3.1 & 46.8 \\ 
   & 262087 & 1 & 3.1 & 49.9 \\ 
   & 274136 & 1 & 3.1 & 53.0 \\ 
   & 280406 & 1 & 3.1 & 56.2 \\ 
   & 291876 & 1 & 3.1 & 59.3 \\ 
   & 331022 & 1 & 3.1 & 62.4 \\ 
   & 439253 & 1 & 3.1 & 65.5 \\ 
   & 469758 & 1 & 3.1 & 68.6 \\ 
   & 477621 & 1 & 3.1 & 71.8 \\ 
   & 482417 & 1 & 3.1 & 74.9 \\ 
   & 502867 & 1 & 3.1 & 78.0 \\ 
   & 539251 & 1 & 3.1 & 81.1 \\ 
   & 586560 & 1 & 3.1 & 84.2 \\ 
   & 838400 & 1 & 3.1 & 87.4 \\ 
   & 848540 & 1 & 3.1 & 90.5 \\ 
   & 901526 & 1 & 3.1 & 93.6 \\ 
   & 914484 & 1 & 3.1 & 96.7 \\ 
   & 1428858 & 1 & 3.1 & 99.8 \\ 
   \hline
 & all & 32 & 99.8 &  \\ 
   \hline
\hline
Población.Total & 43446 & 1 & 3.1 & 3.1 \\ 
   & 44928 & 1 & 3.1 & 6.2 \\ 
   & 77276 & 1 & 3.1 & 9.4 \\ 
   & 78413 & 1 & 3.1 & 12.5 \\ 
   & 78830 & 1 & 3.1 & 15.6 \\ 
   & 115829 & 1 & 3.1 & 18.7 \\ 
   & 270708 & 1 & 3.1 & 21.8 \\ 
   & 358896 & 1 & 3.1 & 25.0 \\ 
   & 375249 & 1 & 3.1 & 28.1 \\ 
   & 496241 & 1 & 3.1 & 31.2 \\ 
   & 515145 & 1 & 3.1 & 34.3 \\ 
   & 575010 & 1 & 3.1 & 37.4 \\ 
   & 877057 & 1 & 3.1 & 40.6 \\ 
   & 967767 & 1 & 3.1 & 43.7 \\ 
   & 993866 & 1 & 3.1 & 46.8 \\ 
   & 1016701 & 1 & 3.1 & 49.9 \\ 
   & 1040157 & 1 & 3.1 & 53.0 \\ 
   & 1065673 & 1 & 3.1 & 56.2 \\ 
   & 1197081 & 1 & 3.1 & 59.3 \\ 
   & 1282063 & 1 & 3.1 & 62.4 \\ 
   & 1298691 & 1 & 3.1 & 65.5 \\ 
   & 1391239 & 1 & 3.1 & 68.6 \\ 
   & 1415933 & 1 & 3.1 & 71.8 \\ 
   & 1419947 & 1 & 3.1 & 74.9 \\ 
   & 1788507 & 1 & 3.1 & 78.0 \\ 
   & 1809116 & 1 & 3.1 & 81.1 \\ 
   & 2090839 & 1 & 3.1 & 84.2 \\ 
   & 2171280 & 1 & 3.1 & 87.4 \\ 
   & 2545924 & 1 & 3.1 & 90.5 \\ 
   & 4756113 & 1 & 3.1 & 93.6 \\ 
   & 6691030 & 1 & 3.1 & 96.7 \\ 
   & 10985285 & 1 & 3.1 & 99.8 \\ 
   \hline
 & all & 32 & 99.8 &  \\ 
   \hline
\hline
\hline
\label{Tfrecuencias}
\end{longtable}
\endgroup

\end{document}
